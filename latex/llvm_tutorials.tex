\documentclass[a4paper,12pt]{book}

\usepackage{graphicx}
\usepackage{ifthen}          %条件参数
\usepackage{xeCJK}           %中文支持
\usepackage{blindtext}
\usepackage{indentfirst}     %首行缩进
\usepackage{nomencl}         % 名词解释 
%\usepackage[nocfg]{nomencl}
\usepackage[acronym]{glossaries}      % 术语表  比较正规的,无二义性的词
% 实话说我也不知道区别在哪里,解释把我搞蒙了。看起来 glossary 更好用一点  
% https://tex.stackexchange.com/questions/63923/difference-between-a-nomenclature-section-and-a-glossary
% https://tex.stackexchange.com/questions/154060/nomenclature-acronym-or-glossary

\makeglossaries
\newglossaryentry{glsy}
{
    name=术语,
    description={Acronyms and terms which are generally unknown or new to common readers.}
}
\newacronym{s2e}{S2E}{Start to End}


\makenomenclature 

\begin{document}

\ifdefined\eng
\else
\newcommand{\eng}{false}
\fi
\newcommand{\engtrue}{true}

\newcommand{\engtitle}[1]{ \ifx\eng\engtrue ({#1}) \fi  }
\newcommand{\engpar}[1]{ \ifx\eng\engtrue \par {#1}  \fi  }

% 术语表加入中文项
\newcommand{\nomchinese}[1]
{
    \renewcommand{\nomentryend}{\hspace*{\fill}\makebox[4cm][l]{#1}}
}

\renewcommand{\nomname}{词汇表} %修改术语表标题的名称。

\nomenclature{LaTeX}{LaTeX is ...}
\nomenclature{A}{A is ...\nomchinese{中文测试}}
\nomenclature{B}{B is ...}
\nomenclature{C}{C is ...}



\author{alan2lin}
\title{
	LLVM 教程翻译 \\
	\large \ifthenelse{\equal{\eng}{true}}{
		Bilingual Edition 双语版 
	}{
		Chinese Edition 纯中文版
	}
}


\date{May 2021}



\frontmatter
\maketitle

\tableofcontents

\mainmatter




\chapter{LLVM 系统入门 \\  \engtitle{Getting Started with the LLVM System} }
\section{概述 \engtitle{Overview}}

\par{欢迎来到 LLVM 项目!}

\engpara{Welcome to the LLVM project!}   

\par{LLVM项目包含多个组件。 该项目的核心本身称为“LLVM”。 它包含处理中间表示层并将其转换为目标文件所需的所有工具,库和头文件。 工具包括汇编器,反汇编器,位码分析器和位码优化器。 它还包含基本的回归测试。}

\engpara{The LLVM project has multiple components. The core of the project is itself called “LLVM”. This contains all of the tools, libraries, and header files needed to process intermediate representations and converts it into object files. Tools include an assembler, disassembler, bitcode analyzer, and bitcode optimizer. It also contains basic regression tests.}

\include{./chapters/chapter02}


aaa \acrlong{s2e} aaaaaaaaaaaaaaa \acrshort{s2e} aaaaaaaaaa  \Gls{glsy}

\printnomenclature
%\printglossaries
\printglossary[type=\acronymtype, title=缩略语表, toctitle=缩略语表]
\printglossary[title=术语表, toctitle=术语表]


\backmatter
% bibliography, glossary and index would go here.



\end{document}
